%
% 000-sesion2.tex
% Presentation Session 2: Tools II - Debugging
%
% Compile to .pdf with LaTeX (pdflatex)
% Requires Beamer package (latex-beamer)
%

\documentclass[xcolor=table]{beamer}
\usetheme{Warsaw}
\beamertemplatenavigationsymbolsempty
\setbeamertemplate{headline}{}
\useoutertheme{infolines}
\usepackage[english]{babel}
\usepackage[utf8]{inputenc}
\usepackage{graphics}
\usepackage{amssymb}
\usepackage{multirow}
\usepackage{xcolor}
\usepackage{framed}
\usepackage{hyperref}

\definecolor{shadecolor}{RGB}{180,180,180}

%%%%%%%%%%%%%%%%%%%%%%%%%%%%%%%%%%%%%%%%%%%%%%%%%%%%%%%%%%%%%%%%
% Python Highlighting
%%%%%%%%%%%%%%%%%%%%%%%%%%%%%%%%%%%%%%%%%%%%%%%%%%%%%%%%%%%%%%%%
\DeclareFixedFont{\ttb}{T1}{txtt}{bx}{n}{10}
\DeclareFixedFont{\ttm}{T1}{txtt}{m}{n}{10}

% Custom colors
\usepackage{color}
\definecolor{deepblue}{rgb}{0,0,0.5}
\definecolor{deepred}{rgb}{0.6,0,0}
\definecolor{deepgreen}{rgb}{0,0.5,0}
\definecolor{grey1}{rgb}{0.5,0.5,0.5}

\usepackage{listings}

% Python style for highlighting
\newcommand\pythonstyle{\lstset{
language=Python,
basicstyle=\ttm\small,
morekeywords={self},
keywordstyle=\ttb\color{deepblue},
emphstyle=\ttb\color{deepred},
stringstyle=\color{deepgreen},
commentstyle=\small\color{grey1}\ttm,
frame=single,
showstringspaces=false
}}

% Python environment
\lstnewenvironment{python}[1][]
{
\pythonstyle
\lstset{#1}
}
{}

% Python for inline
\newcommand\pythoninline[1]{{\pythonstyle\lstinline!#1!}}

\newcommand{\asignatura}{Session 2: Tools II - Debugging}
\newcommand{\grado}{Programming in Network Environments}
\newcommand{\curso}{2025-2026}

\title[\asignatura]{\asignatura}
\subtitle{\grado}
\author{URJC}
\institute{Universidad Rey Juan Carlos}
\date{Course \curso}

\AtBeginSection[]
{
\begin{frame}<beamer>
\begin{center}
{\Huge \insertsection}
\end{center}
\end{frame}
}

\begin{document}

\frame{
\maketitle
}

\begin{frame}
\frametitle{Contents}
\tableofcontents
\end{frame}

%%%%%%%%%%%%%%%%%%%%%%%%%%%%%%%%%%%%%%%%%%%%%%%%%%%%%%%%%%%%%%%%
%%%%%%%%%%%%%%%%%%%%%%%%%%%%%%%%%%%%%%%%%%%%%%%%%%%%%%%%%%%%%%%%
% SECTION 1: Session 1 Review
%%%%%%%%%%%%%%%%%%%%%%%%%%%%%%%%%%%%%%%%%%%%%%%%%%%%%%%%%%%%%%%%
%%%%%%%%%%%%%%%%%%%%%%%%%%%%%%%%%%%%%%%%%%%%%%%%%%%%%%%%%%%%%%%%

\section{Review: Session 1 - Tools I}

\begin{frame}
\frametitle{What We Learned in Session 1}

\textbf{Session 1: Tools I covered}:

\begin{itemize}
\item \textbf{PyCharm IDE}
  \begin{itemize}
  \item Installation and configuration
  \item Creating and running programs
  \item Project management
  \end{itemize}
\vspace{0.2cm}
\item \textbf{GitHub and Git}
  \begin{itemize}
  \item Creating repositories
  \item Push and pull operations
  \item Version control workflow
  \end{itemize}
\vspace{0.2cm}
\item \textbf{First programs}
  \begin{itemize}
  \item hello.py - Hello World
  \item count.py - Print numbers 1 to 20
  \item sum20.py - Sum first 20 integers
  \end{itemize}
\end{itemize}

\end{frame}

%%%%%%%%%%%%%%%%%%%%%%%%%%%%%%%%%%%%%%%%%%%%%%%%%%%%%%%%%%%%%%%%
%%%%%%%%%%%%%%%%%%%%%%%%%%%%%%%%%%%%%%%%%%%%%%%%%%%%%%%%%%%%%%%%
% SECTION 2: Introduction to Debugging
%%%%%%%%%%%%%%%%%%%%%%%%%%%%%%%%%%%%%%%%%%%%%%%%%%%%%%%%%%%%%%%%
%%%%%%%%%%%%%%%%%%%%%%%%%%%%%%%%%%%%%%%%%%%%%%%%%%%%%%%%%%%%%%%%

\section{Introduction to Debugging}

\begin{frame}
\frametitle{Session 2 Goals}

\textbf{In this session we will learn}:

\begin{itemize}
\item How to execute programs \textbf{step by step}
\item Understanding \textbf{breakpoints}
\item Difference between:
  \begin{itemize}
  \item \textbf{Step Over}
  \item \textbf{Step Into}
  \item \textbf{Step Out}
  \end{itemize}
\item Practice debugging with exercises
\end{itemize}

\vspace{0.5cm}

\textbf{Time}: 2 hours

\end{frame}

\begin{frame}
\frametitle{What is Debugging?}

\begin{block}{Bugs and Debugging}
When our programs don't work or generate strange errors, we say they have \textbf{bugs}. The process of finding these errors is called \textbf{debugging}.
\end{block}

\vspace{0.3cm}

\textbf{Why is debugging important?}
\begin{itemize}
\item Writing programs is not difficult
\item Making them work as we want is more complex
\item We need tools to understand what's happening
\item Step by step execution helps us find errors
\end{itemize}

\vspace{0.3cm}

\textbf{PyCharm's integrated debugger} lets us:
\begin{itemize}
\item Execute programs line by line
\item Examine variable values
\item Understand program flow
\end{itemize}

\end{frame}

%%%%%%%%%%%%%%%%%%%%%%%%%%%%%%%%%%%%%%%%%%%%%%%%%%%%%%%%%%%%%%%%
%%%%%%%%%%%%%%%%%%%%%%%%%%%%%%%%%%%%%%%%%%%%%%%%%%%%%%%%%%%%%%%%
% SECTION 3: Step by Step Execution
%%%%%%%%%%%%%%%%%%%%%%%%%%%%%%%%%%%%%%%%%%%%%%%%%%%%%%%%%%%%%%%%
%%%%%%%%%%%%%%%%%%%%%%%%%%%%%%%%%%%%%%%%%%%%%%%%%%%%%%%%%%%%%%%%

\section{Step by Step Execution}

\begin{frame}
\frametitle{Hello World Step by Step}

\textbf{Program}: hello.py (from Session 1)

\vspace{0.3cm}

\textbf{Process}:
\begin{enumerate}
\item Open hello.py in PyCharm
\item Set a \textbf{breakpoint} on the first line
\item Click the \textbf{debug button} (green bug)
\item Use \textbf{Step Over} to execute line by line
\item Watch output in the Console tab
\end{enumerate}

\end{frame}

\begin{frame}
\frametitle{Setting Breakpoints}

\begin{block}{What is a Breakpoint?}
A breakpoint is a marker that tells the debugger where to pause program execution.
\end{block}

\vspace{0.3cm}

\textbf{How to set a breakpoint}:
\begin{itemize}
\item Go to the line where you want to stop
\item Double-click on the right side of the line number
\item A red circle appears
\end{itemize}

\vspace{0.3cm}

\textbf{What happens}:
\begin{itemize}
\item Program executes until it reaches the breakpoint
\item Execution pauses
\item We can examine the current state
\item The line with the breakpoint is highlighted
\end{itemize}

\end{frame}

\begin{frame}
\frametitle{Debug Mode}

\textbf{Starting debug mode}:
\begin{itemize}
\item Set breakpoints
\item Click the green \textbf{bug button} (next to run button)
\item Program starts and pauses at first breakpoint
\end{itemize}

\vspace{0.3cm}

\textbf{Debug panel opens}:
\begin{itemize}
\item \textbf{Debugger tab}: Shows execution controls
\item \textbf{Console tab}: Shows program output
\item \textbf{Variables panel}: Shows current variable values
\item Green dot on bug icon: indicates debug mode active
\end{itemize}

\end{frame}

\begin{frame}
\frametitle{Step Over}

\begin{block}{What is Step Over?}
\textbf{Step Over} executes the current line and moves to the next line in the same scope.
\end{block}

\vspace{0.3cm}

\textbf{How to use}:
\begin{itemize}
\item Press the \textbf{Step Over} button
\item Or press \textbf{F8} key
\end{itemize}

\vspace{0.3cm}

\textbf{What happens}:
\begin{itemize}
\item Current instruction executes
\item Next instruction is highlighted
\item If it's a function call, the function executes completely
\item Execution stops at the next line
\end{itemize}

\end{frame}

\begin{frame}[fragile]
\frametitle{Example: count.py}

\textbf{Program}: Print numbers from 1 to 20

\begin{python}
# Session 1. Exercise 2

for i in range(1, 21):
    print(i, end=' ')
\end{python}

\vspace{0.3cm}

\textbf{Debugging steps}:
\begin{enumerate}
\item Set breakpoint on line 4 (print statement)
\item Click debug button
\item Observe variable \pythoninline{i} in Variables panel
\item Use \textbf{Step Over} to execute print
\item Use \textbf{Resume} (F9) to jump to next breakpoint
\end{enumerate}

\end{frame}

\begin{frame}[fragile]
\frametitle{Example: sum20.py}

\textbf{Program}: Add the 20 first integer numbers

\begin{python}
# Session 1. Exercise 3
res = 0

for i in range(1, 21):
    res += i

print("Total sum: ", res)
\end{python}

\vspace{0.3cm}

\textbf{Debugging steps}:
\begin{enumerate}
\item Set breakpoint on line 5 (\pythoninline{res += i})
\item Watch how \pythoninline{res} and \pythoninline{i} change
\item Use Resume to see incremental changes
\item Final result: \pythoninline{res = 210}
\end{enumerate}

\end{frame}

\begin{frame}
\frametitle{Resume Button}

\begin{block}{What is Resume?}
\textbf{Resume} continues execution until the next breakpoint or program end.
\end{block}

\vspace{0.3cm}

\textbf{How to use}:
\begin{itemize}
\item Press the \textbf{Resume} button (green play icon)
\item Or press \textbf{F9} key
\end{itemize}

\vspace{0.3cm}

\textbf{Useful when}:
\begin{itemize}
\item You're in a loop and want to skip iterations
\item You want to reach the next breakpoint quickly
\item You've examined enough and want to continue
\end{itemize}

\end{frame}

%%%%%%%%%%%%%%%%%%%%%%%%%%%%%%%%%%%%%%%%%%%%%%%%%%%%%%%%%%%%%%%%
%%%%%%%%%%%%%%%%%%%%%%%%%%%%%%%%%%%%%%%%%%%%%%%%%%%%%%%%%%%%%%%%
% SECTION 4: Stepping Into Functions
%%%%%%%%%%%%%%%%%%%%%%%%%%%%%%%%%%%%%%%%%%%%%%%%%%%%%%%%%%%%%%%%
%%%%%%%%%%%%%%%%%%%%%%%%%%%%%%%%%%%%%%%%%%%%%%%%%%%%%%%%%%%%%%%%

\section{Stepping Into Functions}

\begin{frame}[fragile]
\frametitle{Example: sumN.py}

\textbf{Program}: Calculate sum of N first integers using a function

\begin{python}
def sumn(n):
    res = 0
    for i in range(1, n+1):
        res += i
    return res

# Main program
print("Sum of 20 first integers: ", sumn(20))
print("Sum of 100 first integers: ", sumn(100))
\end{python}

\vspace{0.2cm}

\textbf{New folder}: Store in S02/ directory

\end{frame}

\begin{frame}
\frametitle{Step Over vs Step Into}

\textbf{Step Over} (F8):
\begin{itemize}
\item Executes the entire function
\item Returns result without entering function
\item Stops at next line in current scope
\end{itemize}

\vspace{0.3cm}

\textbf{Step Into} (F7):
\begin{itemize}
\item Enters inside the function
\item Stops at first line of function
\item Allows debugging function internals
\item Can see function's local variables
\end{itemize}

\vspace{0.3cm}

\textbf{When to use Step Into}:
\begin{itemize}
\item When you need to debug what happens inside a function
\item To understand function's internal logic
\item To check function variable values
\end{itemize}

\end{frame}

\begin{frame}
\frametitle{Step Out}

\begin{block}{What is Step Out?}
\textbf{Step Out} completes execution of current function and returns to the caller.
\end{block}

\vspace{0.3cm}

\textbf{How to use}:
\begin{itemize}
\item Press the \textbf{Step Out} button
\item Or press \textbf{Shift+F8}
\end{itemize}

\vspace{0.3cm}

\textbf{What happens}:
\begin{itemize}
\item Function completes execution
\item Returns to the line that called the function
\item Stops at the next instruction after function call
\end{itemize}

\vspace{0.3cm}

\textbf{Useful when}:
\begin{itemize}
\item You've seen enough inside the function
\item You want to return to upper level
\item You accidentally stepped into a function
\end{itemize}

\end{frame}

\begin{frame}
\frametitle{Variable Scopes}

\textbf{When stepping into/out of functions}:

\vspace{0.3cm}

\textbf{Inside function scope}:
\begin{itemize}
\item See function's local variables (n, res, i)
\item See function parameters
\item Cannot see main program variables
\end{itemize}

\vspace{0.3cm}

\textbf{Outside function scope} (main program):
\begin{itemize}
\item Local function variables disappear
\item Only see main program variables
\item Variables exist only in their scope
\end{itemize}

\vspace{0.3cm}

\textbf{Example}: In sumN.py
\begin{itemize}
\item Inside sumn(): see n=20, res, i
\item Outside sumn(): those variables don't exist
\end{itemize}

\end{frame}

%%%%%%%%%%%%%%%%%%%%%%%%%%%%%%%%%%%%%%%%%%%%%%%%%%%%%%%%%%%%%%%%
%%%%%%%%%%%%%%%%%%%%%%%%%%%%%%%%%%%%%%%%%%%%%%%%%%%%%%%%%%%%%%%%
% SECTION 5: Exercises
%%%%%%%%%%%%%%%%%%%%%%%%%%%%%%%%%%%%%%%%%%%%%%%%%%%%%%%%%%%%%%%%
%%%%%%%%%%%%%%%%%%%%%%%%%%%%%%%%%%%%%%%%%%%%%%%%%%%%%%%%%%%%%%%%

\section{Exercises}

\begin{frame}
\frametitle{Practice is Key!}

\begin{block}{Learning by Doing}
The only way to learn debugging is by debugging programs!
\end{block}

\vspace{0.3cm}

\textbf{Before starting}:
\begin{itemize}
\item Make sure Session 1 exercises are uploaded to Github
\item Create S02/ folder for Session 2 exercises
\item Practice step by step execution
\end{itemize}

\vspace{0.3cm}

\textbf{Three exercises}:
\begin{enumerate}
\item Fibonacci sequence (basic)
\item Fibonacci function (with Step Into)
\item Fibonacci sum (nested functions)
\end{enumerate}

\end{frame}

\begin{frame}[fragile]
\frametitle{Exercise 1: Fibonacci Sequence}

\textbf{Filename}: S02/fibonacci.py

\vspace{0.3cm}

\textbf{Description}:
\begin{itemize}
\item Write a program (without functions)
\item Print first 11 terms of Fibonacci series
\item Start from 0
\item Execute step by step
\end{itemize}

\vspace{0.3cm}

\textbf{Expected output}:
\begin{verbatim}
0 1 1 2 3 5 8 13 21 34 55
\end{verbatim}

\vspace{0.3cm}

\textbf{Fibonacci reminder}:
\begin{itemize}
\item F(0) = 0, F(1) = 1
\item F(n) = F(n-1) + F(n-2)
\end{itemize}

\end{frame}

\begin{frame}[fragile]
\frametitle{Exercise 2: Fibonacci Function}

\textbf{Filename}: S02/fiboN.py

\vspace{0.3cm}

\textbf{Description}:
\begin{itemize}
\item Convert previous program into function \pythoninline{fibon(n)}
\item Function returns nth Fibonacci term
\item Main program calls fibon() for 5th, 10th, 15th terms
\item Execute step by step
\item Use \textbf{Step Into} to watch function internals
\item Try \textbf{Step Out} button
\end{itemize}

\vspace{0.3cm}

\textbf{Expected output}:
\begin{verbatim}
5th term: 3
10th term: 34
15th term: 377
\end{verbatim}

\end{frame}

\begin{frame}[fragile]
\frametitle{Exercise 3: Fibonacci Sum}

\textbf{Filename}: S02/fibo-sumN.py

\vspace{0.3cm}

\textbf{Description}:
\begin{itemize}
\item Write function \pythoninline{fibosum(n)}
\item Calculates addition of first n Fibonacci terms
\item Main program calls with n=5 and n=10
\item Execute step by step
\item Use \textbf{Step Into} and \textbf{Step Out}
\item Notice: 2 function levels (fibosum calls fibon)
\item Use Step Into \textbf{twice} to enter fibon()
\end{itemize}

\vspace{0.3cm}

\textbf{Expected output}:
\begin{verbatim}
Sum of first 5 terms: 7
Sum of first 10 terms: 88
\end{verbatim}

\end{frame}

%%%%%%%%%%%%%%%%%%%%%%%%%%%%%%%%%%%%%%%%%%%%%%%%%%%%%%%%%%%%%%%%
%%%%%%%%%%%%%%%%%%%%%%%%%%%%%%%%%%%%%%%%%%%%%%%%%%%%%%%%%%%%%%%%
% SECTION 6: Summary
%%%%%%%%%%%%%%%%%%%%%%%%%%%%%%%%%%%%%%%%%%%%%%%%%%%%%%%%%%%%%%%%
%%%%%%%%%%%%%%%%%%%%%%%%%%%%%%%%%%%%%%%%%%%%%%%%%%%%%%%%%%%%%%%%

\section{Summary}

\begin{frame}
\frametitle{Debugging Commands Summary}

\begin{center}
\begin{tabular}{|l|l|l|}
\hline
\textbf{Command} & \textbf{Key} & \textbf{Action} \\
\hline
Step Over & F8 & Execute current line \\
 & & Stay in same scope \\
\hline
Step Into & F7 & Enter function \\
 & & Debug function internals \\
\hline
Step Out & Shift+F8 & Complete function \\
 & & Return to caller \\
\hline
Resume & F9 & Continue to next \\
 & & breakpoint \\
\hline
\end{tabular}
\end{center}

\end{frame}

\begin{frame}
\frametitle{Session Checklist}

\textbf{Make sure you have}:

\begin{itemize}
\item All Session 1 items checked
\item Session-01 folder with:
  \begin{itemize}
  \item count.py
  \item sum20.py
  \end{itemize}
\item Session-02 folder with:
  \begin{itemize}
  \item sumN.py
  \item fibonacci.py (Exercise 1)
  \item fiboN.py (Exercise 2)
  \item fibo-sumN.py (Exercise 3)
  \end{itemize}
\item Everything uploaded to Github
\end{itemize}

\end{frame}

\begin{frame}
\frametitle{Key Takeaways}

\textbf{What we learned}:

\begin{itemize}
\item Setting and using \textbf{breakpoints}
\item \textbf{Step Over}: Execute line in current scope
\item \textbf{Step Into}: Enter function to debug internals
\item \textbf{Step Out}: Return to caller
\item \textbf{Resume}: Continue to next breakpoint
\item Understanding \textbf{variable scopes}
\item Debugging is essential for finding bugs
\end{itemize}

\vspace{0.5cm}

\textbf{Practice debugging regularly!}

\end{frame}

\begin{frame}
\begin{center}
\Huge{Questions?}

\vspace{1cm}

\Large{Happy Debugging!}
\end{center}
\end{frame}

\end{document}
