%
% 000-session7.tex
% Presentation Session 7: Practice 1 - Seq1 Module (OOP)
%
% Compile to .pdf with LaTeX (pdflatex)
% Requires Beamer package (latex-beamer)
%

\documentclass[xcolor=table]{beamer}
\usetheme{Warsaw}
\beamertemplatenavigationsymbolsempty
\setbeamertemplate{headline}{}
\useoutertheme{infolines}
\usepackage[english]{babel}
\usepackage[utf8]{inputenc}
\usepackage{graphics}
\usepackage{amssymb}
\usepackage{multirow}
\usepackage{xcolor}
\usepackage{framed}
\usepackage{hyperref}

\definecolor{shadecolor}{RGB}{180,180,180}

%%%%%%%%%%%%%%%%%%%%%%%%%%%%%%%%%%%%%%%%%%%%%%%%%%%%%%%%%%%%%%%%
% Python Highlighting
%%%%%%%%%%%%%%%%%%%%%%%%%%%%%%%%%%%%%%%%%%%%%%%%%%%%%%%%%%%%%%%%
\DeclareFixedFont{\ttb}{T1}{txtt}{bx}{n}{10}
\DeclareFixedFont{\ttm}{T1}{txtt}{m}{n}{10}

% Custom colors
\usepackage{color}
\definecolor{deepblue}{rgb}{0,0,0.5}
\definecolor{deepred}{rgb}{0.6,0,0}
\definecolor{deepgreen}{rgb}{0,0.5,0}
\definecolor{grey1}{rgb}{0.5,0.5,0.5}

\usepackage{listings}

% Python style for highlighting
\newcommand\pythonstyle{\lstset{
language=Python,
basicstyle=\ttm\small,
morekeywords={self},
keywordstyle=\ttb\color{deepblue},
emphstyle=\ttb\color{deepred},
stringstyle=\color{deepgreen},
commentstyle=\small\color{grey1}\ttm,
frame=single,
showstringspaces=false
}}

% Python environment
\lstnewenvironment{python}[1][]
{
\pythonstyle
\lstset{#1}
}
{}

% Python for inline
\newcommand\pythoninline[1]{{\pythonstyle\lstinline!#1!}}

\newcommand{\asignatura}{Session 7: Practice 1 -- Seq1 Module}
\newcommand{\grado}{Programming in Network Environments}
\newcommand{\curso}{2025-2026}

\title[\asignatura]{\asignatura}
\subtitle{\grado}
\author{URJC}
\institute{GSyC}
\date{Course \curso}

\AtBeginSection[]
{
\begin{frame}<beamer>
\begin{center}
{\Huge \insertsection}
\end{center}
\end{frame}
}

\begin{document}

\frame{
\maketitle
}

\begin{frame}
\frametitle{Contents}
\tableofcontents
\end{frame}

%%%%%%%%%%%%%%%%%%%%%%%%%%%%%%%%%%%%%%%%%%%%%%%%%%%%%%%%%%%%%%%%
%%%%%%%%%%%%%%%%%%%%%%%%%%%%%%%%%%%%%%%%%%%%%%%%%%%%%%%%%%%%%%%%
% SECTION 1: Session 6 Review
%%%%%%%%%%%%%%%%%%%%%%%%%%%%%%%%%%%%%%%%%%%%%%%%%%%%%%%%%%%%%%%%
%%%%%%%%%%%%%%%%%%%%%%%%%%%%%%%%%%%%%%%%%%%%%%%%%%%%%%%%%%%%%%%%

\section{Review: Session 6 -- Object Oriented Programming}

\begin{frame}
\frametitle{What We Learned in Session 6}

\textbf{Session 6: Object Oriented Programming}:

\begin{itemize}
\item \textbf{From functions to objects}
  \begin{itemize}
  \item Procedural: data and functions are \textbf{separated}
  \item OOP: data (\textbf{attributes}) and functions (\textbf{methods}) grouped in \textbf{objects}
  \end{itemize}
\vspace{0.15cm}
\item \textbf{Classes}: templates for creating objects
  \begin{itemize}
  \item \pythoninline{__init__(self, strbases)}: initialization method
  \item \pythoninline{__str__(self)}: controls how object is printed
  \item \pythoninline{len(self)}: custom method example
  \end{itemize}
\vspace{0.15cm}
\item \textbf{Inheritance}: \pythoninline{class Gene(Seq)}
  \begin{itemize}
  \item Child class inherits methods from parent
  \item Can add new methods or \textbf{override} existing ones
  \item Use \pythoninline{super().__init__()} to call parent constructor
  \end{itemize}
\vspace{0.15cm}
\item \textbf{termcolor}: installed an external library for colored output
\end{itemize}

\end{frame}

%%%%%%%%%%%%%%%%%%%%%%%%%%%%%%%%%%%%%%%%%%%%%%%%%%%%%%%%%%%%%%%%
%%%%%%%%%%%%%%%%%%%%%%%%%%%%%%%%%%%%%%%%%%%%%%%%%%%%%%%%%%%%%%%%
% SECTION 2: Session 7 Overview
%%%%%%%%%%%%%%%%%%%%%%%%%%%%%%%%%%%%%%%%%%%%%%%%%%%%%%%%%%%%%%%%
%%%%%%%%%%%%%%%%%%%%%%%%%%%%%%%%%%%%%%%%%%%%%%%%%%%%%%%%%%%%%%%%

\section{Session 7: Practice 1 -- Seq1 Module}

\begin{frame}
\frametitle{Session 7 Goals}

\textbf{In this session we will}:

\begin{itemize}
\item Create the \textbf{Seq Class} in a module called \textbf{Seq1.py}
\item Apply \textbf{Object Oriented Programming} to DNA sequences
\item Similar to Seq0 (Practice 0) but using an \textbf{OOP approach}
\item Handle three types of sequences: \textbf{Null}, \textbf{Valid}, \textbf{Invalid}
\end{itemize}

\vspace{0.5cm}

\textbf{Time}: 2 hours

\vspace{0.3cm}

\textbf{Important}: Finish all previous practices before starting!

\end{frame}

\begin{frame}
\frametitle{The Seq Class: Methods Overview}

\begin{center}
\begin{tabular}{|l|l|l|}
\hline
\textbf{Method} & \textbf{Parameters} & \textbf{Description} \\
\hline
\pythoninline{len()} & None & Total number of bases \\
\hline
\pythoninline{count_base(base)} & base: char & Count of a given base \\
\hline
\pythoninline{count()} & None & Dict with all base counts \\
\hline
\pythoninline{reverse()} & None & Reverse sequence \\
\hline
\pythoninline{complement()} & None & Complement sequence \\
\hline
\pythoninline{read_fasta(file)} & filename: str & Read FASTA into object \\
\hline
\end{tabular}
\end{center}

\vspace{0.3cm}

\textbf{Special methods} (already known from S6):
\begin{itemize}
\item \pythoninline{__init__(self, strbases=None)}: initialize the object
\item \pythoninline{__str__(self)}: print the object as a sequence
\end{itemize}

\end{frame}

\begin{frame}[fragile]
\frametitle{Three Types of Sequences (Ex. 1--3)}

The Seq class must handle \textbf{three types} of sequences:

\vspace{0.3cm}

\begin{itemize}
\item \textbf{Null}: created with no arguments -- \pythoninline{Seq()}
  \begin{itemize}
  \item Prints: ``NULL sequence created''
  \item Uses \textbf{optional parameter}: \pythoninline{def __init__(self, strbases=None)}
  \end{itemize}
\vspace{0.2cm}
\item \textbf{Valid}: only contains A, T, C, G -- \pythoninline{Seq("ACTGA")}
  \begin{itemize}
  \item Prints: ``New sequence created!''
  \end{itemize}
\vspace{0.2cm}
\item \textbf{Invalid}: contains other characters -- \pythoninline{Seq("Invalid")}
  \begin{itemize}
  \item Prints: ``INVALID sequence!''
  \item Stored as ``ERROR''
  \end{itemize}
\end{itemize}

\vspace{0.3cm}

\textbf{Exercises 1--3}: Create the Seq1 module, import it, and test the three sequence types.

\end{frame}

\begin{frame}[fragile]
\frametitle{Methods: len, count\_base, count (Ex. 4--6)}

\textbf{Exercise 4}: \pythoninline{len(self)}
\begin{itemize}
\item Returns total number of bases
\item Null or Invalid sequences $\rightarrow$ length is \textbf{0}
\end{itemize}

\vspace{0.2cm}

\textbf{Exercise 5}: \pythoninline{count_base(self, base)}
\begin{itemize}
\item Returns count of a specific base (A, C, T, or G)
\item Null or Invalid sequences $\rightarrow$ returns \textbf{0}
\end{itemize}

\vspace{0.2cm}

\textbf{Exercise 6}: \pythoninline{count(self)}
\begin{itemize}
\item Returns a \textbf{dictionary} with all base counts
\item Example: \pythoninline{\{'A': 2, 'T': 1, 'C': 1, 'G': 1\}}
\item Null or Invalid $\rightarrow$ all values are \textbf{0}
\end{itemize}

\vspace{0.3cm}

\textbf{Pattern}: Always check sequence type first -- return 0 or empty result for Null/Invalid.

\end{frame}

\begin{frame}[fragile]
\frametitle{Methods: reverse, complement, read\_fasta (Ex. 7--9)}

\textbf{Exercise 7}: \pythoninline{reverse(self)}
\begin{itemize}
\item Returns the reversed sequence string
\item ``ACTGA'' $\rightarrow$ ``AGTCA''
\item Null/Invalid $\rightarrow$ returns ``NULL'' or ``ERROR''
\end{itemize}

\vspace{0.2cm}

\textbf{Exercise 8}: \pythoninline{complement(self)}
\begin{itemize}
\item A $\leftrightarrow$ T, C $\leftrightarrow$ G
\item ``ACTGA'' $\rightarrow$ ``TGACT''
\item Null/Invalid $\rightarrow$ returns ``NULL'' or ``ERROR''
\end{itemize}

\vspace{0.2cm}

\textbf{Exercise 9}: \pythoninline{read_fasta(self, filename)}
\begin{itemize}
\item Creates a \textbf{Null} sequence first, then reads from file
\item Populates the object with the FASTA sequence
\end{itemize}

\begin{python}
s = Seq()                # Create Null sequence
s.read_fasta(FILENAME)   # Load from file
\end{python}

\end{frame}

\begin{frame}
\frametitle{Exercise 10 and Deliverables}

\textbf{Exercise 10}: Processing the genes
\begin{itemize}
\item Find the \textbf{most frequent base} in each gene
\item Same as Practice 0 Ex. 8 but using \textbf{Seq Class}
\item Genes: U5, ADA, FRAT1, FXN, RNU6\_269P
\end{itemize}

\vspace{0.5cm}

\textbf{P01 folder should contain}:

\begin{itemize}
\item \textbf{Seq1.py} -- The module with the Seq Class
\item \textbf{e1.py} through \textbf{e10.py} -- One program per exercise
\end{itemize}

\vspace{0.3cm}

\textbf{Remember}:
\begin{itemize}
\item Push everything to your remote Github repository
\item Make sure all exercises produce the expected output
\item Check all items in the session checklist
\end{itemize}

\end{frame}

%%%%%%%%%%%%%%%%%%%%%%%%%%%%%%%%%%%%%%%%%%%%%%%%%%%%%%%%%%%%%%%%
%%%%%%%%%%%%%%%%%%%%%%%%%%%%%%%%%%%%%%%%%%%%%%%%%%%%%%%%%%%%%%%%
% SECTION 3: End
%%%%%%%%%%%%%%%%%%%%%%%%%%%%%%%%%%%%%%%%%%%%%%%%%%%%%%%%%%%%%%%%
%%%%%%%%%%%%%%%%%%%%%%%%%%%%%%%%%%%%%%%%%%%%%%%%%%%%%%%%%%%%%%%%

\section{Let's start!}

\begin{frame}
\begin{center}
\Huge{Questions?}

\vspace{1cm}

\Large{Time to code!}
\end{center}
\end{frame}

\end{document}
